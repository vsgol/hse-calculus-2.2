\Subsection{Билет 74: Неравенство Коши. Целые функции. Примеры. Теорема Лиувилля. Основная теорема алгебры.}

\begin{theorem}[Неравенство Коши]\thmslashn
	
	$f \in H(R \mathbb{D})\;\;r < R\;\; f(z) =  \sum\limits_{n = 0}^{\infty} a_n z^n$ 
	
	Тогда $|a_n| \leqslant \frac{M(r)}{r^n}$, где $M(r) = \max\limits_{|z| = r} |f(z)|$
	
\end{theorem}

\begin{proof}\thmslashn
	
	 $|a_n| =  \abs{\frac{1}{2\pi i}\int\limits_{|\xi| = r} \frac{f(\xi)}{\xi^{n+1}}} \leqslant \frac{1}{2\pi} \cdot 2\pi r \cdot \max\limits_{|\xi| = r}\abs{\frac{f(\xi)}{\xi^{n+1}}} = r \max\limits_{|\xi| = r}\abs{f(\xi)} \cdot \frac{1}{r^{n+1}} = \frac{M(r)}{r^n}$
	
\end{proof}

\begin{definition}
		Целая функция $f:\CC \to \CC$ и $f\in H(\CC)$
\end{definition}

\begin{example}\thmslashn
	
	\begin{enumerate}
		\item 
		$e^{z}$
		\item
		Многочлены
		\item
		$\sin z$ и $\cos z$, тк $\sin = \frac{e^{iz} - e^{-iz}}{zi};\;\;\; \cos = \frac{e^{iz} + e^{-iz}}{z}$
		\item
		$\ch z = \frac{e^{z} + e^{-z}}{z};\;\;\; \sh z = \frac{e^{z} - e^{-z}}{z}$
	\end{enumerate}
	
\end{example}

\begin{theorem}[Лиувилля]\thmslashn
	
	$f$ -- целая и ограниченная функция $\Rightarrow$ $f$ -- константа
	
\end{theorem}

\begin{proof}\thmslashn
	
	$f(z) = \sum\limits_{n = 0}^{\infty} a_n z^n \qquad |f(z)| \leqslant M$
	
	По неравенству Коши $|a_n| \leqslant \frac{M(r)}{r^n} \leqslant \frac{M}{r^n} \to 0 \Rightarrow a_n = 0$ при $n\not = 0 \Rightarrow f$ -- постоянная
	
\end{proof}

\begin{theorem}[Основная теорема алгебра]\thmslashn
	
	$P$ -- многочлен, $P \not=$ const $\Rightarrow P$ имеет корень
	
\end{theorem}


\begin{proof}\thmslashn
	
	От противного. Пусть $P(z) \not = 0 \;\; \forall z \in \CC$
	
	Тогда $f(z) = \frac{1}{P(z)} \in H(\CC)$
	
	Докажем, что $|f(z)|$ -- ограничен
	
	$P(z) = z^n + a_{n-1}z^{n-1} + a_{n-2}z^{n-2} + \ldots + a_0$
	
	Пусть $|z| \geqslant R = 1 + |a_{n-1}| + |a_{n-2}| + \ldots + |a_0|$	
	
	Оценим теперь многочлен
	
	$|P(z)| \geqslant |z|^n - |a_{n-1}z^{n-1}| - |a_{n-2}z^{n-2}| - \ldots - |a_0| \geqslant$
	
	
	$|z|^n - |a_{n-1}||z^{n-1}| - |a_{n-2}z^{n-1}| - \ldots - |a_0||z^{n-1}| = $
	
	
	$|z|^n - |z^{n-1}|(R-1) = |z|^{n-1} (|z| - R + 1) \geqslant R^{n-1} \geqslant 1 $
	
	Вне круга $|z| \leqslant R \quad |P(z)| \geqslant 1$
	
	В круге $|z| \geqslant R \quad |P(z)|$ непрерывно $\Rightarrow$ достигается минимум 
	
	$\Rightarrow |P(z)| \geqslant |P(z_0)| > 0 \Rightarrow |P(z)| \geqslant m = \min(1, |P(z_0)|)$
	
	$\Rightarrow |f(z)| \leqslant 1/m \Rightarrow f = $ const по теорему Лиувилля $\Rightarrow P =$ const
	
\end{proof}

\begin{consequence}\thmslashn
	
	Если $\deg P = n$, то $P(z) = c(z - z_1)(z-z_2)\ldots(z-z_n)$
	
\end{consequence}

\begin{proof}
	Индукция
\end{proof}
