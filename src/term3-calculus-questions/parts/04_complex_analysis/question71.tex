\Subsection{Билет 71: Индекс кривой относительно точки. Интегральная формула Коши.}

\begin{definition}[Индекс пути отностительно точки]\thmslashn
	
	$\gamma$ -- замкнутая кривая, не проходящая через $0$
	
	$r(t)$ и $\phi(t)$ -- ее задание в полярных координатах $r(t) > 0$
	
	$r, \phi : [a,b] \to \R$ непрерывны
	
	$\Ind(\gamma, 0) := \frac{\phi(b) - \phi(b)}{2\pi} \in \Z$
	
\end{definition}

\begin{remark}\thmslashn
	
	Это можно понимать немного по другому
	
	Если у нас есть замкнутый путь, не проходящий через точку $O$, то если мы проведем из этой точки луч, то каждое пересечение по часовой стрелке дает $-1$, а против -- $1$
	
\end{remark}

\begin{theorem}\thmslashn
	
	$\int\limits_\gamma \frac{dz}{z} = 2\pi i \, \Ind(\gamma, 0)$, если $\gamma$ не проходит через $0$
	
\end{theorem}

\begin{proof}\thmslashn
	
	$r(t)$ и $\phi(t)$ параметризация $\gamma$ в полярных координатах 
	
	$z = re^{i\phi}\;\; dz = d(r(t)e^{i\phi(t)}) = r'(t) e^{i\phi(t)} + r(t)i\phi'(t)e^{i\phi(t)}$

	$\frac{dz}{z} = \frac{r'(t)}{r(t)} + i\phi'(t)$
	
	$\int\limits_\gamma \frac{dz}{z} = \int\limits_a^b  \frac{r'(t)}{r(t)} + i\phi'(t)\,dt = \ln r(t)\Big|_a^b + i\phi(t)\Big|_a^b = i (\phi(b) - \phi(a)) = 2\pi i \, \Ind(\gamma, 0)$
	
	$r(a) = r(b)$, т.к. кривая замкнута

\end{proof}

\begin{consequence}\thmslashn
	
	$\Ind(\gamma, a) = \frac{1}{2\pi i}\int\limits_\gamma \frac{dz}{z-a}$
	
\end{consequence}

\begin{theorem}[интегральная формула Коши]\thmslashn
	
	$f\in H(\Omega) \;\; a\in \Omega$
	
	$\gamma$ -- стягивающийся путь, не проходящей через $a$. 
	
	Тогда $\frac{1}{2\pi i}\int\limits_\gamma \frac{f(z)}{z-a}\,dz = f(a)\cdot \Ind(\gamma, a)$
	
\end{theorem}

\begin{proof}\thmslashn
	
	$
	g(z) :=
	\begin{cases}
		\frac{f(z) - f(a)}{z-a}, & \text{при } z \not = a \\
		f'(a), & \text{при } z = a
	\end{cases}
	$
	
	Тогда $g \in C(\Omega)$ и $g\in H(\Omega/\{a\})$
	
	$\Rightarrow g(z)dz$ -- локально точная форма $\Rightarrow \int\limits_\gamma g(z)\,dz = 0 \Rightarrow \int\limits_\gamma \frac{f(z) - f(a)}{z-a}\,dz = 0 = \int\limits_\gamma \frac{f(z)}{z-a}\,dz - f(a) \int\limits_\gamma \frac{dz}{z-a} = \int\limits_\gamma \frac{f(z)}{z-a}\,dz - f(a) \Ind(\gamma, a)$
\end{proof}

\begin{example}\thmslashn
	
	У нас есть круг, с границей $\gamma$. 
	
	$f$ голоморфна в окрестности круга
	
	$\int\limits_\gamma \frac{f(z)}{z-a}\,dz = 
	\begin{cases}
        0, & a \text{ вне круга, так как } Ind(\gamma, a) = 0 \text{, сделали 0 оборотов}\\
	2\pi i f(a), & \text{внутри круга}
	\end{cases}$
	
    В частности $f(a) = \frac{1}{2\pi i} \int\limits_{|z-a| = r} \frac{f(z)}{z-a}\,dz$

    (кажется, здесь на лекции была опечатка, и не было коэффициента $\frac{1}{2\pi i}$)
\end{example}