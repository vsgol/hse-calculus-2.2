\Subsection{Билет 73: Теорема Мореры. Следствие. Вторая версия интегральной формулы Коши. Условия, равносильные голоморфности}

\begin{theorem}[Мореры]\thmslashn
	
	$f\in C(\omega)$ и $f(z)dz$ локально точная в $\Omega$
	
	Тогда $f \in H(\Omega)$ 
	
\end{theorem}

\begin{proof}\thmslashn
	
	Возьмем $a\in \Omega$
	
	У нее несть окрестность, в которой $f(z)dz$ имеет первообразную, т.е. существует $F$, т.ч. существует $F$, т.ч. $F' = f$
	
	Тогда $F$ голоморфна в этой окрестности $\Rightarrow f = F'$ -- голоморфна в этой окрестности $\Rightarrow f$ голоморфна в точке $a \Rightarrow f \in H(\Omega)$    
	
\end{proof}

\begin{consequence}\thmslashn
	
	$f \in C(\Omega)$, $\Delta$ -- прямая параллельная оси координнат 
	
	Если $f \in H(\Omega /\Delta)$, то $f \in H(\Omega)$
	
\end{consequence}

\begin{proof}\thmslashn
	
	По следствию теоремы Коши $f\in C(\Omega) \& f\in H(\Omega /\Delta) \Rightarrow f(z)dz$ локально точна в $\Omega \Rightarrow f \in H(\Omega)$ по теореме Морера
		
\end{proof}

\begin{theorem}[интегральная формула Коши]

	$f\in H(\Omega), \;\; K\subset \Omega\;\; K$ -- компакт с кусочно-гладкой границей
	
	Тогда $\int\limits_{\partial K} f(z)\,dz = 0$ и если $a \in \Int K$, то $\int\limits_{\partial K} \frac{f(z)}{z-a}\,dz = 2\pi i f(a)$

\end{theorem}

\begin{proof}\thmslashn
		
	\begin{enumerate}
		\item 
		$\int\limits_{\partial K} f(z)\,dx + f(z)dy = \int\limits_{K}\left( \frac{\partial if(z)}{\partial y} - \frac{\partial f(z)}{\partial x} \right)\,dx\,dy = 0$
		
		Первый переход это формула Грина, а второй верен, тк это условие Коши-Римана 
		
		\item
		Выберем кружочек вокруг точки $a$ так, чтобы он с границей полностью лежал в $\Int K$
		
		$\tilde{K} = K \setminus(a + r\mathbb{D})$ -- компакт
		
		$0 = \int\limits_{\partial \tilde{K}} \frac{f(z)}{z-a}\,dz = \int\limits_{\partial K} \frac{f(z)}{z-a}\,dz + \int\limits_{\text{окр}} \frac{f(z)}{z-a}\,dz =  \int\limits_{\partial K} \frac{f(z)}{z-a}\,dz - \int\limits_{|z-a| = r} \frac{f(z)}{z-a}\,d = \int\limits_{\partial K} \frac{f(z)}{z-a}\,dz - 2\pi i f(a)$
		
		Во втором переходе у нас изменился знак перед вторым интегралом, тк мы изменили направление обхода, а последний переход это интегральная формула Коши.
	\end{enumerate}
	
\end{proof}

\begin{exerc}
	$f\in C(\Cl \mathbb{D})$ и $f \in H(\mathbb{D})$, $a \in \mathbb{D}$
	
	Доказать, что $\int\limits_{\partial K} \frac{f(z)}{z-a}\,dz = 2\pi i f(a)$
\end{exerc}

\begin{theorem}\thmslashn
	
	$f:\Omega \to \CC$ 
	
	Следующие условия равносильны
	
	\begin{enumerate}[1)]
		\item 
		$f\in H(\Omega)$
		\item
		$f$ аналитична в $\Omega$
		\item
		$f$ локально имеет первообразную
		\item
		$f(z)dz$ локально точная форма и $f \in C(\Omega)$
		\item
		$f(z)dz$ замкнутая форма с непрерывными частными производными 
		\item
		$f\in C(\Omega)$ и интеграл $\int f(z)\,dz$ по любому достаточно малому прямоугольнику, со сторонами, параллельными осям координат
		 
	\end{enumerate}
	
\end{theorem}

\begin{proof}\thmslashn
	\begin{enumerate}
		\item[1)$\Leftrightarrow$2)]
		Это следствие из теоремы 4.6
		\item[1)$\Rightarrow$4)]
		Это теорема Коши
		\item[4)$\Rightarrow$1)]
		Это теорема Морера
		\item[3)$\Leftrightarrow$4)]
		Мне вообще кажется, что это тождественные определения
		\item[5)$\Rightarrow$4)]
		Это общее свойство форм (следствие леммы Пуанкаре)
		\item[2)+4)$\Rightarrow$5)]
		Теорема 3.4
		\item[6)$\Rightarrow$4)]
		Это теорема 3.2
		\item[5)$\Rightarrow$6)]
		Формула Грина
		
	\end{enumerate}
	
\end{proof}