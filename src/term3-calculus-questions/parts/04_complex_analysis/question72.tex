\Subsection{Билет 72: Аналитичность голоморфной функции. Следствия}

\begin{theorem}\thmslashn
	
	$f\in H(r \mathbb(D)) \Rightarrow f$ аналитична в $r \mathbb{D}$ 
	
\end{theorem}

\begin{proof}\thmslashn
	
	$0 < r_1 < r_2 < r$
	
	Запишем интегральную формулу Коши для круга радиуса $r_2$ 
	
	$f(z) = \frac{1}{2\pi i} \int\limits_{|\xi| = r_2} \frac{f(\xi)}{\xi-z}\,d\xi$
	
	$\frac{1}{\xi - z} = \frac{1}{\xi} \cdot \frac{1}{1 - z/\xi} = \frac{1}{\xi} \left( 1 + \frac{z}{\xi} + \frac{z^2}{\xi^2} + \ldots \right)$ это равномерно сходящийся ряд, при $|z| \leqslant r_1$
	
	$= \frac{1}{2\pi i} \int\limits_{|\xi| = r_2} f(\xi) \sum\limits_{n = 0}^{\infty} \frac{z^n}{\xi^{n+1}}\,d\xi = \frac{1}{2\pi i}\sum\limits_{n = 0}^{\infty} z^n \int\limits_{|\xi| = r_2} \frac{f(\xi)}{\xi^{n+1}} = \sum\limits_{n = 0}^{\infty} a_n z^n$
	
	где $a_n =  \frac{1}{2\pi i}\int\limits_{|\xi| = r_2} \frac{f(\xi)}{\xi^{n+1}}$
	
	$f(z) = \sum\limits_{n = 0}^{\infty} a_n z^n$ такое разложение есть для всех  $z$ и оно не зависит от $r_2$, тк $k$ коэффициент это $\frac{f^{(k)}}{k!} \Rightarrow$ от того, что мы поменяли $r_2$ не может ничего изменится.
	
	
\end{proof}

\begin{consequence}\thmslashn
	
	\begin{enumerate}
		\item 
		Если $f(z) = \sum\limits_{n = 0}^{\infty} a_n z^n$ сходится в круге $r \mathbb{D}$, то 
		$a_n = \frac{1}{2\pi i}\int\limits_{|\xi| = r_2} \frac{f(\xi)}{\xi^{n+1}}$, где $0 < r_1 < r$
		
		\item
		$f:\Omega \to \CC \;\; f$ -- голоморфна в $\Omega \Leftrightarrow f$ -- аналитична в $\Omega$
		
		\begin{proof}\thmslashn
			
			\begin{enumerate}
				\item ["$\Rightarrow$"]
				$z_0+r\mathbb{D}\subset \Omega \Rightarrow f$ голоморфна в $z_0+r\mathbb{D} \Rightarrow f$ раскладывается в ряд, сходящийся в $z_0+r\mathbb{D}$, по степеням $z - z_0$
				
				\item ["$\Leftarrow$"]
				$f$ -- аналитична в точке $z_0 \Rightarrow f$ дифференцируема в точке $z_0$
			\end{enumerate}
		\end{proof} 
		
		\item
		$f\in H(\Omega) \Rightarrow f$ бесконечно дифференцируема в $\Omega$
		
		\item
		$f \in H(\Omega) \Rightarrow f' \in H(\Omega)$
		
		\item
		$f \in H(\Omega) \Rightarrow \Re f$ и $\Im f$ -- гармонические функции
		
		\begin{remark}\thmslashn
			
			$f:\Omega \to \R$ гармоническая, если 
			
			$\frac{\partial^2 f}{\partial x_1^2} + \frac{\partial^2 f}{\partial x_2^2} + \ldots + \frac{\partial^2 f}{\partial x_n^2} \equiv 0$ во всех точках из $\Omega$
		\end{remark}
		
		\begin{proof}\thmslashn
			
			Условие Коши-Римана $\frac{\partial \Re f}{\partial x} = \frac{\partial \Im f}{\partial y}$ и $\frac{\partial \Re f}{\partial y} = -\frac{\partial \Im f}{\partial x}$
			
			$\frac{\partial^2 \Re f}{\partial x^2} = \frac{\partial}{\partial x} \left(\frac{\partial \Im f}{\partial y} \right) = \frac{\partial^2 \Im f}{\partial x \partial y} = \frac{\partial^2 \Im f}{\partial y \partial x} = -\frac{\partial^2 \Re f}{\partial y^2}$
			
		\end{proof}
	
	\end{enumerate}
	
\end{consequence}

\begin{remark}\thmslashn
	
	Если $P$ -- гармоническая функция в $\Omega \subset R^2$, то существует единственная функция с точностью до константы гармоническая функция 
	
	$Q:\Omega \to \R$, т.ч. $P + iQ \in H(\Omega)$
	
\end{remark}

\begin{exerc}
	Доказать это 
\end{exerc}

