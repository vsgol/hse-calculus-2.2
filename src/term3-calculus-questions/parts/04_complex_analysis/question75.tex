\Subsection{Билет 75: Теорема единственности голоморфной функции (с производными). Следствие}

\begin{lemma}\thmslashn
	
	$\Omega$ -- область в метрическом пространстве $E \subset \Omega \;\; E \not = \emptyset$
	
	$E$ открыто в $\Omega$ и $E$ замкнуто в $\Omega$
	
	Тогда $E = \Omega$
	
\end{lemma}

\begin{proof}\thmslashn

	Пусть $\Omega \setminus E \not= \emptyset$
	
	Возьмем $a \in E, b\in \Omega \setminus E$ и соединим из кривой $\gamma:[\alpha, \beta] \to \Omega \quad \gamma(\alpha) = a, \gamma(\beta) = b$
	
	Рассмотрим $\gamma^{-1}(E) \subset [\alpha, \beta]$	открытое множество (прообраз открытого)
	
	$\gamma^{-1}(\Omega \setminus E) \subset [\alpha, \beta] $ открытое множество
	
	$[\alpha, \beta] \setminus \gamma^{-1}(E) = \gamma^{-1}(\Omega \setminus E) \Rightarrow \gamma^{-1}(E)$ замкнуто в $[\alpha, \beta]$
	
	Рассмотрим $s:= \sup \gamma^{-1}(E) \quad s \in \gamma^{-1}(E)$ т.к. $\gamma^{-1}(E)$ -- замкнуто
	
    $\Rightarrow s < \beta\qquad$ (ведь $\beta \in \gamma^{-1}(\Omega \setminus E)$)

    Но $\gamma^{-1}(E)$ открыто $\Rightarrow$ $\exists \delta > 0$, т.ч. $(s-\delta, s+ \delta) \subset \gamma^{-1}(E)$
	
	Это противоречит тому, что $s = \sup$
	
\end{proof}

\begin{theorem}[единственности]\thmslashn
	
	$f\in H(\Omega) \;\;z_0 \in \Omega$
	
	Следующие условия равносильны 
	
	\begin{enumerate}[1)]
		\item 
		$f^{(n)}(z_0) = 0 \;\; \forall n \geqslant 0$
		\item
		$f\equiv 0 $ в окрестности точки $z_0$
		\item
		$f \equiv 0$ в $\Omega$
	\end{enumerate}
	
\end{theorem}

\begin{proof}\thmslashn
	
	\begin{enumerate}
		\item[1)$\Rightarrow$2)] 
		$f\in H(\Omega) \;\; z_0 + r\mathbb{D} \subset \Omega \Rightarrow f$ раскладывается в ряд тейлора в круге $z_0 + r\mathbb{D} \Rightarrow$
		
		$f(z) = \sum\limits_{n= 0}^{\infty} \frac{f^{(n)}(z_0)}{n!}(z - z_0)^n = 0$ при $|z-z_0| < r$
		\item[2)$\Rightarrow$1)]
		Очевидно
		\item[3)$\Rightarrow$1)]
		Очевидно
		\item[2)$\Rightarrow$3)]
		$E = \{z \in \Omega: \text{ существует окр-ть точки }z\text{, т.ч. } f\equiv 0\text{ в этой окр-ти} \}$
		
		$z_0 \in E \Rightarrow E \not = \emptyset$
		
		$E$ -- открытое, так как у каждой точки есть окрестность, где 0, так что точки из этой окрестности для неё и будут шариком
		
		Пусть $z_n \in E$ и $z_n \to z_* \in \Omega$
		
		Надо доказать, что $z_* \in E$
		
        $f^{(k)}(z_n) = 0\;\; \forall k, n \;\; f^{(k)}(z_n) \to f^{(k)}(z_*) \Rightarrow$
		
		$f^{(k)}(z_*) = 0 \Rightarrow z_* \in E$, тк 1)$\Rightarrow$2)
		
		$\Rightarrow E$ -- замкнуто. 

        (предельный переход в производных возможен, так как производные непрерывны из-за бесконечной дифференцируемости голоморфной функции)
		
		Следовательно по лемме $E = \Omega \Rightarrow f\equiv 0$ в $\Omega$
		
	\end{enumerate}
		
\end{proof}

\begin{consequence}\thmslashn

	\begin{enumerate}
		\item 
		$f, g \in H(\Omega)$ и $f = g$ в окрестности точки $z_0 \Rightarrow f= g$ в $\Omega$
		\item
		$f, g \in H(\Omega)$ и $f^{(n)}(z_0) = g^{(n)}(z_0)\;\; \forall n \Rightarrow f = g$ в $\Omega$ 
	\end{enumerate}
	
\end{consequence}

\begin{proof}
	Подставим в теорему $f - g$
\end{proof}