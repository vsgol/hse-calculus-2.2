\Subsection{Билет 49: Признак Вейерштрасса. Следствие. Пример.}

\begin{theorem}[Вейерштрасса]\thmslashn
	
	$f, g: [a, +\infty)\times T \to \R$ интегрируемы на $[a, b]$ $\forall t \in T$ и $\forall b > a$
	
	$\abs{f(x, t)} \leqslant g(x, t) \;\; \forall t \in T \;\; \forall x > a$ и $\int\limits_{a}^{+\infty} g(x, t)\,dx$ сх-ся равномерно
	
	Тогда $\int\limits_{a}^{+\infty} f(x, t)\,dx$ сх-ся равномерно
	
\end{theorem}

\begin{proof}\thmslashn
	
	Критерий Коши 
	
	$\int\limits_{a}^{+\infty} g(x, t)\,dx$ равномерно сх-ся $\Rightarrow$ 
	
	$\forall \varepsilon > 0\;\; \exists B > a\;\; \forall b, c > b\;\; \forall t \in T \quad \abs{\int\limits_{b}^{c}g(x, t)\,dx} < \varepsilon$
	
	$\abs{\int\limits_{b}^{c}f(x, t)\,dx} \leqslant \int\limits_{b}^{c}\abs{f(x, t)}\,dx \leqslant \abs{\int\limits_{b}^{c}g(x, t)\,dx} < \varepsilon \Rightarrow$ по критерию Коши $\int\limits_{a}^{+\infty}f(x, t)\,dx$ равномерно сх-ся
	
\end{proof}

\begin{consequence}\thmslashn
	
	Если $\abs{f(x, t)}\leqslant g(x) \;\; \forall t \in T \;\; \forall x > a$ и $\int\limits_{a}^{+\infty} g(x)\,dx$ сходится
	
	Тогда $\int\limits_{a}^{+\infty}f(x, t)\,dx$ равномерно сх-ся
	
\end{consequence}

\begin{example}\thmslashn
	
	$\int\limits_{0}^{+\infty} \dfrac{\cos(xt)}{1 + x^2}\,dx$ сх-ся равномерно
	
	$g(x) = \dfrac{1}{1 + x^2} \;\; \int\limits_{0}^{+\infty} \dfrac{1}{1 + x^2}\,dx$ -- сх-ся $\abs{\dfrac{\cos(xt)}{1 + x^2}} \leqslant g(x)$
	
\end{example}

\begin{remark}\thmslashn
	
	Если у $\abs{f(x, t)}$ есть суммируемая мажоранта, то сх-ть $\int\limits_{a}^{+\infty}f(x, t)\,dx$ равномерная.
	
\end{remark}
