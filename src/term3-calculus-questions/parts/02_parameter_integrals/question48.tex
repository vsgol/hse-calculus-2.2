\Subsection{Билет 48: Равномерная сходимость несобственных интегралов с параметром. Критерий Коши. Следствие. Примеры равномерно и неравномерно сходящихся интегралов.}

\begin{definition}
	Равномерная сходимость интеграла
	
	$\forall \varepsilon > 0 \;\; \exists B > a \;\; \forall b > B \;\; \forall t \in T \quad \abs{\int\limits_b^{+\infty} f(x, t)\, dx} < \varepsilon$
	
\end{definition}

\begin{remark}\thmslashn

    $F_b(t) = \int\limits_a^b f(x, t) \, dx$
	
	$\int\limits_a^{+\infty} f(x, t)\, dx$  -- равномерно сх-ся $\Leftrightarrow$ $F_b(t)  \toto{b\to \infty} F(t)$ равномерно по $t \in T$
	
	$\forall \varepsilon > 0\;\; \exists B \;\; \forall b > B \;\; \forall t\in T \quad \abs{F_b(t) - F(t)} < \varepsilon$
	
\end{remark}

\begin{proof}\thmslashn
	
	$\abs{F_b(t) - F(t)} = \abs{\int\limits_b^{+\infty} f(x, t)\, dx}$	ну значит получили определение

\end{proof}

\begin{example}\thmslashn
	
	$\int\limits_b^{+\infty} e^{-tx}\, dx \leqslant \int\limits_b^{+\infty} e^{-t_0x}\, dx  = \dfrac{e^{-t_0x}}{-t_0}\Big|_{x = b}^{x = +\infty} = \dfrac{e^{-t_0b}}{t_0}$

	Но при $t > 0$	уже нет равномерной сходимости $\int\limits_b^{+\infty} e^{-tx}\, dx = \dfrac{e^{-tb}}{t} = \dfrac{b}{e}$ при $t = \dfrac{1}{b}$ мы для любого $b$ можем сделать интеграл большим $\Rightarrow$ сходится неравномерно.
\end{example}

\begin{theorem}[Критерий Коши равномерной сходимости интегралов]\thmslashn
	
	$\int\limits_a^{+\infty} f(x, t)\, dx$  -- равномерно сх-ся $\Leftrightarrow$ 
	
	$\forall \varepsilon > 0\;\; \exists B > a\;\; \forall b, c > B\;\; \forall t \in T \quad \abs{\int\limits_{b}^{c}f(x, t)\,dx} < \varepsilon$
	
\end{theorem}

\begin{proof}\thmslashn
	
	$\int\limits_a^{+\infty} f(x, t)\, dx$  -- равномерно сх-ся $\Leftrightarrow$ $F_b(t)  \toto{t\to t_0} F(t)$ равномерно по $t \in T$
	
	\begin{enumerate}
		\item ["$\Rightarrow$"]
		$
		\left.
		\begin{aligned}
		\forall \varepsilon > 0\;\; \exists b > a\;\; &\forall b > B\;\; \forall t \in T \quad |F_b - F| < \varepsilon \\
		&\forall c > B\;\; \forall t \in T \quad |F_c- F| < \varepsilon
		\end{aligned}
		\right\rbrace \Rightarrow |F_c - F_b| = \abs{\int\limits_{b}^{c}f(x, t)\,dx} < 2\varepsilon
		$
		
		\item["$\Leftarrow$"]
		По критерию Коши для фиксированного $t \in T \Rightarrow \int\limits_{a}^{+\infty}f(x, t)\,dx$  сходится
		
		$\forall \varepsilon > 0\;\; \exists b > a\;\; \forall b,c > B\;\; \forall t \in T\quad \abs{\int\limits_{b}^{c}f(x, t)\,dx} < \varepsilon$
		
		$\abs{\int\limits_{b}^{c}f(x, t)\,dx} \underset{c\to \infty}\to \abs{\int\limits_{b}^{+\infty}f(x, t)\,dx} \leqslant \varepsilon$
		
		Но это определение равномерной сходимости
		
	\end{enumerate}
	
\end{proof}

\begin{consequence}\thmslashn
	
	$f : [a, +\infty) \times [c, d] \to \R$ непрерывна и $\int\limits_{b}^{+\infty}f(x, t)\,dx$ равномерно сходится на $(c, d)$
	
	Тогда он равномерно сходится на $[c,d]$
	
\end{consequence}


\begin{proof}\thmslashn
	
	$\forall \varepsilon > 0\;\; \exists b > a\;\; \forall b,b' > B\;\; \forall t \in (c, d)\quad \abs{\int\limits_{b}^{b'}f(x, t)\,dx} < \varepsilon$
	
	$f$ непрерывна на $[b,b'] \times [c, d] \Rightarrow  \abs{\int\limits_{b}^{b'}f(x, t)\,dx} \to  \abs{\int\limits_{b}^{b'}f(x, c)\,dx} \leqslant \varepsilon$
	
	Аналогично с $d$ $\abs{\int\limits_{b}^{b'}f(x, t)\,dx} \to  \abs{\int\limits_{b}^{b'}f(x, d)\,dx} \leqslant \varepsilon$
	
	$\Rightarrow \forall t \in [c, d] \quad \abs{\int\limits_{b}^{b'}f(x, t)\,dx} \leqslant \varepsilon \Rightarrow \abs{\int\limits_{a}^{\infty}f(x, t)\,dx}$ равномерно сходится на $[c, d]$ по критерию Коши
	
\end{proof}

\begin{consequence}\thmslashn
	
	$f : [a, +\infty) \times [c, d] \to \R$ непрерывна и $\int\limits_{b}^{+\infty}f(x, t)\,dx$ расходится при $t = c$ или $t = d$
	
	Тогда $\int\limits_{b}^{+\infty}f(x, t)\,dx$ не может равномерно сх-ся на $(c, d)$
	
\end{consequence}

\begin{example}\thmslashn
	
	$\int\limits_{0}^{+\infty} e^{-tx^2}\,dx \quad$ $
	\left.
	\begin{aligned}
	&\text{сх-ся при }t > 0 \\
	&\text{расх-ся при } t = 0
	\end{aligned}
	\right\rbrace \Rightarrow$ на $(0, +\infty)$ нет равномерной сх-ти
	
	При $t \geqslant t_0 > 0$ есть равномерная сх-ть 
	
	$\int\limits_{b}^{+\infty} e^{-tx^2}\,dx \leqslant \int\limits_{b}^{+\infty} e^{-t_0x^2}\,dx \leqslant \int\limits_{b}^{+\infty} e^{-t_0x}\,dx = \dfrac{e^{-t_0x}}{-t_0}\Big|_{x=b}^{x = +\infty} = \dfrac{e^{-t_0b}}{t_0} \underset{b \to \infty}\to 0$
	
	$b \geqslant 1$
	
\end{example}
