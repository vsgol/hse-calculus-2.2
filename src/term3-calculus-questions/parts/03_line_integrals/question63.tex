\Subsection{Билет 63: Точные, локально точные и замкнутые формы. Связь между локальной точностью и замкнутостью. Лемма Пуанкаре (доказательство для $\R^2$. Пример, показывающий, что из замкнутости не следует точность. Следствия.}

\begin{definition}\thmslashn
	
	Форма называется точной, если у нее есть перообразная
	
	Форма локально точная, если у каждой точки найдется окр-ть, в которой есть своя первообразная
	
	Форма $\omega = f_1 dx_1 + f_2dx_2 + \ldots + f_ndx_n$ замкнутая, если $\frac{\partial f_i}{\partial x_k} = \frac{\partial f_k}{\partial x_i} \;\;\forall i, k$
	
\end{definition}

\begin{remark}\thmslashn
	
	\begin{enumerate}
		\item 
		Точность $\Rightarrow$ локальная точность
		
		\item
		Интеграл от точной формы о замкнутому контуру $ = 0$
		
	\end{enumerate}
	
\end{remark}

\begin{theorem}\thmslashn

	Если коэфф. формы непрерывно дифференцируем, то из локальной точности следует замкнутость 

\end{theorem}

\begin{proof}\thmslashn
	
	Возьмем точку $x$ и ее окрестность $U$, в которой $\omega = dF$
	
	$\Rightarrow f_k = \frac{\partial F}{\partial x_k} \Rightarrow \frac{\partial f_k}{\partial x_i} = \frac{\partial^2 F}{\partial x_i\partial x_k} = \frac{\partial^2 F}{\partial x_k\partial x_i} = \frac{\partial f_i}{\partial x_k} \Rightarrow  \omega$  -- замкнута
	
\end{proof}

\begin{lemma}[Пуанкаре]\thmslashn
	
	$\Omega$ -- выпуклая область и коэфф. $\omega$ непрерывно дифференцируемы. Тогда если $\omega$ замкнута, то $\omega$ точная
	
\end{lemma}

\begin{proof}\thmslashn
	
	Только для $\R^2$. Надо доказать, что из замкнутости следует существование первообразной. Для этого достаточно проверить, что $\int\limits_\gamma \omega = 0$ для любой замкнутой кривой $\gamma$
	
	$\int\limits_\gamma P\,dx + Q\,dy = \int\limits_\Omega \left( \frac{\partial Q}{\partial x}-\frac{\partial P}{\partial y} \right) \,dx\,dy = 0$
	
	$ \frac{\partial Q}{\partial x}-\frac{\partial P}{\partial y} = 0$ по замкнутости
	
	Если выпуклости не будет, то на области с дыркой будут проблемы (формула $ \frac{\partial Q}{\partial x}-\frac{\partial P}{\partial y}$ не определена в дырке)
	
\end{proof}

\begin{example}\thmslashn
	
	$\omega = \frac{x \,dy - y\,dx}{x^2 + y^2} = d\left( \arctg \frac{x}{y} \right)$
	
	$\frac{\partial \left(\frac{x}{x^2+y^2}\right)}{\partial x} = \frac{\partial \left(-\frac{y}{x^2+y^2}\right)}{\partial y}$ Замкнутая, но она не является точной
	
	$\int\limits_{\text{един.\\окр.}} \omega = \int\limits_0^{2\pi} \cos t \,d\sin t - \sin t \,d\cos t = \int\limits_0^{2\pi} \,dt = 2\pi$
	
	Проблема в точке $(0, 0)$
\end{example}


\begin{consequence}\thmslashn
	
	\begin{enumerate}
		\item 
		Замкнутая форма с непрерывно дифференцируемыми коэффициентами в любом открытом шаре имеет первообразную
		
		\item
		Замкнутая форма с непрерывно дифференцируемыми коэффициентами локально точна
	
	\end{enumerate}
	
\end{consequence}
