\Subsection{Билет 59: Определение и свойства интеграла по длине дуги (равенства и неравенства).}

\begin{definition}\thmslashn
	
	Интеграл $\RomanNumeral{1}$ рода (интеграл по длине дуги)
	
	$\gamma$ -- гладкая кривая в $\R^n$
	
	$\int\limits_{\gamma} f \,ds := \int\limits_{a}^{b} f(\gamma(t)) \norm{\gamma'(t)} \,dt$
	
	$\gamma: [a, b] \to R^{n}$
	
\end{definition}

\begin{properties}\thmslashn
	
	\begin{enumerate}[1.]
		\item 
		Не зависит от параметризации
		
		\begin{proof}\thmslashn
			
			$\tilde{\gamma} = \gamma \circ \tau$, где $\tau: [c, d] \to [a, b]$, строго возрастает, гладкое $\tau(c) = a, \;\;\tau(d) = b$
			
			$\int\limits_{\tilde{\gamma}} f \,ds = \int\limits_{c}^{d} f(\gamma(\tau(t))) \norm{(\gamma\circ\tau)'(u)} \,du = \int\limits_{c}^{d} f(\gamma(\tau(t))) \tau'(u) \norm{\gamma'(\tau(u))} \,du \underset{t = \tau(u)}= \int\limits_{a}^{b} f(\gamma(t)) \norm{\gamma'(t)} \,dt = \int\limits_{\gamma} f \,ds$
			
			$(\gamma\circ \tau)'(u) = \gamma'(\tau(u))\cdot \tau'(u)$	
			
		\end{proof}
		
		\item
		Интеграл не зависит от направления 
		
		\begin{proof}\thmslashn

			$\tau(c) = b,\;\; \tau(d) = a$, тогда

			
			$\int\limits_{\tilde{\gamma}} f \,ds = \int\limits_{c}^{d} f(\gamma(\tau(t))) \norm{(\gamma\circ\tau)'(u)} \,du = \int\limits_{c}^{d} f(\gamma(\tau(t))) (-\tau'(u)) \norm{\gamma'(\tau(u))} \,du \underset{t = \tau(u)}= -\int\limits_{b}^{a} f(\gamma(t)) \norm{\gamma'(t)} \,dt = \int\limits_{\gamma} f \,ds$
			

		\end{proof}
	
		\item
		Линейность интеграла $\int\limits_{\gamma} (\alpha f + \beta g) \,ds = \alpha \int\limits_{\gamma} f \,ds + \beta \int\limits_{\gamma} g \,ds$
		
		\begin{proof}\thmslashn
			
			$\int\limits_{\gamma} (\alpha f + \beta g) \,ds = \int\limits_{a}^{b} (\alpha f(\gamma(t)) + \beta g(\gamma(t)) ) \norm{\gamma'(t)} \,dt = \alpha \int\limits_{a}^{b} f(\gamma(t)) \norm{\gamma'(t)} \,dt + \beta \int\limits_{a}^{b} f(\gamma(t)) \norm{\gamma'(t)} \,dt = \alpha \int\limits_{\gamma} f \,ds + \beta \int\limits_{\gamma} g \,ds$
	
		\end{proof}

		\item 
		Аддитивность по кривой $\gamma = \gamma_1 \sqcup \gamma_2 \;\; \int\limits_{\gamma} f\,ds = \int\limits_{\gamma_1} f\,ds + \int\limits_{\gamma_2} f\,ds$
		
		\begin{proof}\thmslashn
			
			$\gamma:[a,b \to \R^n \;\;\; c\in [a, b] \;\;\; \gamma_1 = \gamma\Big|_{[a, c]}\;\;\;\gamma_2 = \gamma\Big|_{[c, b]}$
			
			$\int\limits_{\gamma} f\,ds = \int\limits_{a}^{b} f(\gamma(t)) \norm{\gamma'(t)} \,dt = \int\limits_{a}^{c} + \int\limits_{c}^{b} = \int\limits_{\gamma_1} f\,ds + \int\limits_{\gamma_2} f\,ds $
			
		\end{proof}

		\item
		$\int\limits_{\gamma}\,ds = l(\gamma)$ -- длина кривой
		
		\item
		Если $f \leqslant g$ на $\gamma$, то $\int\limits_{\gamma} f\,ds \leqslant \int\limits_{\gamma}g\,ds$
		
		
		\begin{proof}\thmslashn
			
			$\int\limits_{\gamma} f\,ds = \int\limits_{a}^{b} f(\gamma(t)) \norm{\gamma'(t)} \,dt \leqslant \int\limits_{a}^{b} g(\gamma(t)) \norm{\gamma'(t)} \,dt = \int\limits_{\gamma} g\,ds$
			
		\end{proof}
	
		\item
		$\abs{\int\limits_{\gamma} f\,ds} \leqslant \int\limits_{\gamma}\abs{f}\,ds \leqslant l(\gamma) \cdot \max\abs{f}$
	
		\begin{proof}\thmslashn
			
			$\abs{\int\limits_{\gamma} f\,ds} = \abs{\int\limits_{a}^{b} f(\gamma(t)) \norm{\gamma'(t)} \,dt} \leqslant \int\limits_{a}^{b} \abs{f(\gamma(t))} \norm{\gamma'(t)} \,dt \leqslant \max\abs{f}\cdot \int\limits_{a}^{b} \norm{\gamma'(t)}\,dt = l(\gamma) \cdot \max\abs{f}$
			
		\end{proof}
		
	\end{enumerate}

\end{properties}

\begin{remark}
	Можно определить на кусочно-гладких кривых и все свойства сохраняются.
\end{remark}

\begin{exerc}\thmslashn
	
	$\int\limits_{\gamma} f \,ds = \lim \sum\limits_{k=1}^n f(\gamma(\xi_k)) \cdot l(\gamma\Big|_{[t_{k-1}, t_k]})$
	
\end{exerc}