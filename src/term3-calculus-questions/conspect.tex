% Содержит кучу стандартных настроек.
% Настоятельно рекомендуется для использования

\input{core}
\usepackage{epigraph}
\usepackage{import}

%\enablecode % Включает поддержку кода.
%\lstset {
%  style=supercpp% uncomment to use c++ everywhere.
%}
%\lstset {
%  texcl=true %Включить, если хотите писать русские комментарии в коде.
%}
\enablemath % добавляет новые определения, например: \N, \divby

\begin{document}
	% \newcommand\CustomTitle{Здесь можно создать кастомные заголовки в шапке}
	\gdef\CourseName{Билеты по матану} % Обязательно
	\author{\ldots} % Обязательно
	% \gdef\ShortCourseName{Алгебра} % Для другого название в шапке, не обязательно
	% \gdef\LaconicFooter{YES} % Минималистичный футер, только номера страниц.
	% \gdef\NoTitlePage{YES} % Отключить главную страницу.
	
	\makegood
	
	% Если конспект очень большой, то осмысленно разбить его на куски.
	% Можно сохранить отдельную часть в отдельный .tex,
	% а затем написать \input{part}, просто подставить в данное место part.tex
	\newcommand\load[1]{\import{parts/#1/}{00_main}}
	\load{01_lebesgue_integration}
	\load{02_parameter_integrals}
	\load{03_line_integrals}
	\load{04_complex_analysis}
\end{document}